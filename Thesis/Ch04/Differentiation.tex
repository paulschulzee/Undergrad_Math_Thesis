\section{Differentiation}
\subsection{Definition}
One of the primary motivations for infinitesimal calculus is that it allows use to access the more intuitive, less roundabout conceptualizations of derivatives and integrals. Unlike integrals, which still take some work to define nonstandardly, the nonstandard derivative is almost exactly what we would first guess it to be.

Let $\Delta x$ be a nonzero infinitesimal, and let $\Delta f (x, \Delta x) = f(x + \Delta x) - f(x)$. Intuitively, $\Delta x$ is an infinitesimal change in $x$, and $\Delta f$ is the corresponding change in $f$ caused by ``moving'' $\Delta x$ along the $x$-axis. Then we have:

\[ f'(x) = \frac{\Delta f(x, \Delta x)}{\Delta x} \]
One small problem: we'd like $f'$ to be a real-valued function on the reals. Luckily, we have a tool to do that:

\[ f'(x) = \stp{\frac{\Delta f(x, \Delta x)}{\Delta x}} \]
And we have our definition. Well, this might also not be well-defined: we get around \textit{that} problem by definition. We only consider $f'(x)$ to exist when $\stp{\frac{\Delta f(x, \Delta x)}{\Delta x}}$ is the same for \textit{any} nonzero infinitesimal $\Delta x$. In this case, we say that $f$ is \textit{differentiable} at $x$. Note that this definition of differentiability and the derivative is equivalent to the standard definition: see Thm 8.1.1 in \cite{goldblatt1998}.               

\begin{defn}
    Given a function $f: \reals \to \reals$ and a real number $x$, we say that $f$ is \textit{differnetiable} at $x$ if there is some constant $f'(x)$ such that, for any nonzero infinitesimal $\Delta x$,
    \[f'(x) = \stp{\frac{f(x + \Delta x) - f(x)}{\Delta x}}\]
\end{defn}

\subsection{Simple Proofs Using Infinitesimals}
A number of proofs of basic calculus results can be easily accomplished by infinitesimals. Perhaps most striking is the chain rule.

\begin{thm}[Chain Rule]\label{ChainRule}
    Given differentiable functions $f, g: \reals \to \reals$, 
    \[(f \circ g)'(x) = f'(g(x)) \cdot g'(x)\]
\end{thm}

\begin{proof}[Proof by Goldblatt]
    Let $\Delta x$ be any nonzero infinitesimal, and let $\Delta g = g(x + \Delta x) - g(x)$. If $\Delta g = 0$, then $g(x + \Delta x) = g(x)$, so $(f \circ g)'(x) = \stp{\frac{f(g(x + \Delta x)) - f(g(x))}{\Delta x}} = 0$ and clearly $g'(x) = \stp{\Delta g / \Delta x} = 0$, so we are done. If $\Delta g \neq 0$, then since $g'(x) = \stp{\Delta g / \Delta x}$ is defined, we conclude $\Delta g / \Delta x$ is bounded and (since $\Delta x$ is infinitesimal) that $\Delta g$ is infinitesimal. Thus
\begin{align*}
    (f \circ g)'(x) &= \stp{\frac{f(g(x + \Delta x)) - f(g(x))}{\Delta x}} \\
        &= \stp{\frac{f(g(x) + \Delta g) - f(g(x))}{\Delta g} \cdot \frac{\Delta g}{\Delta x}} \\
        &= \stp{\frac{f(g(x) + \Delta g) - f(g(x))}{\Delta g}} \cdot \stp{\frac{\Delta g}{\Delta x}} \\
        &= f'(g(x)) \cdot g'(x)
\end{align*}
\end{proof}

\subsection{Partial Derivatives}
Goldblatt doesn't talk about partial derivatives, but we can easily enough extend our infinitesimal definition of derivative.

\begin{defn}
    If $f: \reals^n \to \reals$, then we define
    \[f_{x_k}(b_1, b_2, \ldots, b_n) = \stp{\frac{f(b_1, \ldots, b_k + \Delta x, \ldots, b_n) - f(b_1, \ldots, b_k, \ldots, b_n)}{\Delta x}} \]
    for any infinitesimal $\Delta x$. This is only defined when it doesn't depend on our choice of $\Delta x$.
\end{defn}

This gets complicated when we want to deal with repeated partial derivatives. Say $f: \reals^2 \to reals$, and denote the inputs of $f$ by $f(x, y)$. Then we might want to write

\begin{align*}
    f_{yx}(a, b) &= \stp{\frac{f_y(a + \Delta x, b) - f_y(a, b)}{\Delta x}} \\
        &= \stp{\frac{\stp{\frac{f(a + \Delta x, b + \Delta y) - f(a + \Delta x, b)}{\Delta y}} - \stp{\frac{f(a, b + \Delta y) - f(a, b)}{\Delta y}}}{\Delta x}}
\end{align*}

which would allow us to easily get $f_{yx}(a, b)= f_{xy}(a, b)$. However, this isn't right. Firstly, the numerator is the difference of two real numbers, and so is real, which would mean $f_{yx}(a, b)$ is either $0$ or undefined (as a real divided by an infinitesimal is either $0$ or unbounded). The mistake here is that $f_y(a + \Delta x, b) \neq \stp{\frac{f(a + \Delta x, b + \Delta y) - f(a + \Delta x)}{\Delta y}}$. Since in this case $f_y$ is taking a nonreal input, we have to use the extension $\hr{f_y}(a + \Delta x, b)$. If $\Delta x = [\Delta x_n]$, then this is $[f_y(a + \Delta x_n, b)] = \left[\stp{\frac{f(a + \Delta x_n, b + \Delta y) - f(a + \Delta x_n, b)}{\Delta y}}\right]$. This is the equivalence class of a sequence of real numbers, but it needn't be a real number itself.

\begin{thm}
    Let $f: \reals^n \to \reals$. Let $\hat{f}_{x_k}$ denote the standard definition of the partial derivative, namely
    \[ \hat{f}_{x_k}(b_1, \ldots, b_n) = \lim_{h \to 0} \frac{f(b_1, \ldots, b_k + h, \ldots, b_n) - f(b_1, \ldots, b_k, \ldots, b_n)}{h} \]
    Then $\hat{f}_{x_k}(b_1, \ldots, b_n)$ exists iff $f_{x_k}(b_1, \ldots, b_n)$ does, and if they both exist $\hat{f}_{x_k}(b_1, \ldots, b_n) = f_{x_k}(b_1, \ldots, b_n)$.
\end{thm}

\begin{proof}
    Let $g: \reals \to \reals$ be defined by $g(x) = f(b_1, \ldots, b_{k-1}, x, b_{k+1}, \ldots, b_n)$. Let $\hat{g}'(x)$ denote the standard derivative of $g$ and $g'(x)$ denote the nonstandard derivative of $g$. Clearly $\hat{g}'(x) = \hat{f}_{x_k}(x)$ and $g'(x) = f_{x_k}(x)$, and by Thm 8.1.1 in \cite{goldblatt1998} we have $\hat{g}'(b_k)$ is defined iff $g'(b_k)$, and $\hat{g}'(b_k) = g'(b_k)$ if both exist, so we are done.
\end{proof}

