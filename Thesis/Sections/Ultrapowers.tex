\section{Ultrafilters, Ultrapowers, \texorpdfstring{$\hreals$}{*R}}
Our goal is to formulate a set of numbers, the hyperreals, denoted $\hreals$. We want this set to include both the real numbers $\reals$ \textit{and} some number of infinitesimal elements, elements that are ``infinitely close to'' a given real number $r$ but still not \textit{equal} to $r$. To do this, we will take an \textit{ultrapower} of $\reals$. This ultrapower will be a quotient ring of $\reals^\nats$, the ring of sequences of real numbers, divided by an equivalence relation. With our equivalence relation we hope to capture the idea that two sequences have the same values ``\textit{almost} everywhere.'' And to do this, we will need an \textit{ultrafilter} on $\nats$.

\subsection{The Ultrafilter}
The idea is that we will define an \textit{ultrafilter} $\mathcal{F} \subseteq \mathcal{P}(\nats)$ that includes all of the ``very large'' subsets of $\nats$, and then we will write $\langle r_n \rangle \equiv \langle s_n \rangle$ when $\{n \in \nats \sthat r_n = s_n\} \in \mathcal{F}$. Here are the criteria we will use:

\begin{defn}[\protect{\cite[18]{goldblatt1998}}]
    $\mathcal{F} \subseteq \nats$ is a (non-principal) ultrafilter on $\nats$ if:
    \begin{itemize}
        \item whenever $A, B \in \mathcal{F}$, we have $A \cap B \in \mathcal{F}$, and
        \item whenever $A \in \mathcal{F}$ and $A \subseteq B$, we have $B \in \mathcal{F}$, and
        \item for any $A \in \mathcal{P}(\nats)$, \textit{either} $A$ or $A^c = \mathcal{P}(\nats) - A$ are in $\mathcal{F}$, and
        \item no finite set is in $\mathcal{F}$.
    \end{itemize}
\end{defn}

A subset of $\mathcal{P}(\nats)$ that satisfies the first two bullets is a \textit{filter}. For instance, $\mathcal{P}(\nats)$ is a filter on $\nats$, as is $\emptyset$. A \textit{proper} filter is one that does not contain $\emptyset$ (as if $\emptyset \in \mathcal{F}$, then $\mathcal{P}(\nats) = \mathcal{F}$ by the second bullet point). Stricly speaking, any proper filter that satisfies the third bullet point is an ultrafilter, but we will henceforth use ``ultrafilter'' to refer only to filters that satisfy all four bullets. 

Call a set \textit{cofinite} if its complement is finite. Every cofinite subset of $\nats$ is in $\mathcal{F}$, by the second and third bullets. 

If we think of $\mathcal{F}$ as being the subsets of $\nats$ that include ``almost all'' of the natural numbers, then these make intuitive sense. If $A$ and $B$ both include ``almost all'' of the natural numbers, then surely $A \cap B$ does too---``basically no'' elements are in $A - B$ or $B - A$ since ``basically no'' elements are outside $A$ or $B$. If $A$ includes ``almost all'' of $\nats$, and $A \subseteq B$, then surely $B$ includes ``almost all'' of $\nats$ too. Etc.

Of course, there are some unintuitive things about $\mathcal{F}$: it contains either the set of even numbers $2\nats$ or the set of odd numbers $2\nats + 1$, but not both, by the third bullet point, for example. 

\begin{thm}[\protect{\cite[Corollary~2.6.2]{goldblatt1998}}]
    There is at least one ultrafilter $\mathcal{F}$ on $\nats$.
\end{thm}

\begin{proof}[Proof adapted from \protect{\cite[20-21]{goldblatt1998}}]
    Let $\mathcal{F}^{co} \subset \mathcal{P}(\nats)$ denote the collection of cofinite subsets of $\nats$. Let $P$ denote the collection of all proper filters on $\nats$ that include $\mathcal{F}^{co}$. Since $\mathcal{F}^{co}$ is itself a filter, $P \neq \emptyset$. 

    $P$ is partially ordered by $\subseteq$: our approach is to apply Zorn's Lemma. Let $T \subset P$ be totally ordered by $\subseteq$. Then $\cup T$ is clearly an upper bound of $T$, but we need to show that $\cup T$ is a filter (this is \cite[Example~2.4(4)]{goldblatt1998}, but is not proven). If $A, B \in \cup T$, then $A \in T_1$ and $B \in T_2$ for some $T_1, T_2 \in T$. Since $T$ is totally ordered by $\subseteq$, either $T_1 \subseteq T_2$ or $T_2 \subseteq T_1$. If $T_1 \subseteq T_2$, then $A \in T_2$, and so since $T_2$ is a filter $A \cap B \in T_2$ and thus $A \cap B \in \cup T$. The proof is similar if $T_2 \subseteq T_1$. Similarly, if $A \in \cup T$ and $A \subseteq B$, then $A \in T_1 \in T$ and so $B \in T_1$ since $T_1$ is  filter, and so $B \in \cup T$.

    So, by Zorn's Lemma, $P$ has a maximal element---call it $\mathcal{F}$. We want to show $\mathcal{F}$ is an ultrafilter (this is \cite[Exercise~2.5(6)]{goldblatt1998}). By the definition of $P$, $\mathcal{F}$ is a proper filter. Since $\mathcal{F}^{co} \subseteq \mathcal{F}$, if we can show that for every $A \subseteq \nats$ either $A$ or $A^c$ is in $\mathcal{F}$ we will be done.

    Take $A \subseteq \nats$. We cannot have $A \in \mathcal{F}$ and $A^c \in \mathcal{F}$, for then we'd have $A \cap A^c = \emptyset \in \mathcal{F}$, which is impossible since $\mathcal{F}$ is proper. Now, assume for a contradiciton that $A \notin \mathcal{F}$ and $A^c \notin \mathcal{F}$. We will show that $\mathcal{F}$ can be extended by adding $A$ or $A^c$, showing that $\mathcal{F}$ is not a maximal element of $P$ and obtaining our contradiction. 

    Let $\mathcal{F}'$ be the filter obtained by adding to $\mathcal{F}$ the set $A$, the intersection $A \cap B$ for any $B \in \mathcal{F}$, and any superset of any of those. Note that $A = A \cap \nats$ and $\nats \in \mathcal{F}$. We will show that $\mathcal{F}' \in P$. First, for any $B \in \mathcal{F}$, we have $B \nsubseteq A^c$ (as $A^c \notin \mathcal{F}$) and so $A \cap B \neq \emptyset$, showing $\mathcal{F}'$ is proper. Next, if $X, Y \in \mathcal{F}$, where $A \cap B \subseteq X$ and $A \cap C \subseteq Y$, then $(A \cap B) \cap (A \cap C) = A \cap (B \cap C) \subseteq X \cap Y$, with $B \cap C \in \mathcal{F}$, so $X \cap Y \in \mathcal{F}'$. Similarly, if $X \in \mathcal{F}'$ and $X \subseteq Y$, where $A \cap B \subseteq X$, then $A \cap B \subseteq Y$ and so $Y \in \mathcal{F}'$.   

    
\end{proof}

