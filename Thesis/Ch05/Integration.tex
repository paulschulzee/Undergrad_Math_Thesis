\section{Integration}
\subsection{Hyperfinite Sets}
\begin{defn}
    An internal set $A = [A_n]$ is \textit{hyperfinite} if every $A_n$ is finite.
\end{defn}

The hyperfinite sets are ``internally finite.'' They share a lot of properties with finite sets.

\begin{thm}\label{HyperfiniteTransfer}
    If $\varphi(A)$ holds for every finite $A \subseteq \reals$, then $\hr{\varphi}(X)$ holds for every hyperfinite $X \subseteq \hreals$.
\end{thm}

\begin{proof}
    Say $X = [A_n]$, with each $A_n$ finite. Then $[[\varphi(A_n)]] = \nats$, and so by transfer $\hr{\varphi}(X)$.
\end{proof}

This lets us easily get a lot of nice properties about hyperfinite sets. For instance, $(\exists x \in A)(\forall y \in A) (x \geq y)$ ensures that every hyperfinite set has a maximum element.

\subsection{Hyperfinite Sums}
For any finite set $A_n$ and function $f_n: \reals \to \reals$, we can easily define the sum $\sum_{x \in A_n} f_n(x)$. This is, after all, just a sum of a finite collection of numbers. Using this, however, we can easily extend our summation to hyperfinite sets:

\begin{defn}
    If $A = [A_n]$ is a hyperfinite set, and $f = [f_n]$ is an internal function, we define the \textit{hyperfinite sum}
    \[
    \sum_{x \in [A_n]} f(x) = \left[\ \ \sum_{\mathclap{x \in A_n}} f_n(x) \:\right].
    \] 
\end{defn}

\subsection{Integrability \& Integrals}
If $f$ is a function bounded on $[a, b]$ and $P = \{x_0 = a, x_1, \ldots, x_n = b\}$ a partition on $[a, b]$, let $M_i$ and $m_i$ be the supremum and infimum respectively of $f$ on $[x_{i-1}, x_i]$, and $\Delta x_i = x_i - x_{i-1}$. We then define:
\begin{itemize}
    \item The upper Reimann sum $U_a^b(f, P) = \sum_{i=1}^n M_i\Delta x_i$
    \item The lower Reimann sum $L_a^b(f, P) =\sum_{i=1}^n m_i\Delta x_i$
    \item The ordinary Riemann sum $S_a^b(f, P) = \sum_{i=1}^n f(x_{i-1})\Delta x_i$
\end{itemize}
For any positive real $\Delta x$, let $P_{\Delta x} = \{a, a+\frac{b-a}{\Delta x}, a + 2\frac{b-a}{\Delta x}, \ldots, a + n\frac{b-a}{\Delta x}, b\}$. Now, for any positive real $\Delta x$, let $U_a^b(f, \Delta x) = U_a^b(f, P_{\Delta x})$. Similarly for $L_a^b(f, \Delta x)$ and $S_a^b(f, \Delta x)$.

A function $f: \reals \to \reals$ is \textit{integrable} on $[a, b]$ if $L_a^b(f, \Delta x) \simeq U_a^b(f, \Delta x)$ for every infinitesimal $\Delta x$. This is equivalent to the standard definition of integrability \cite[110]{goldblatt1998}.

If $f$ is integrable on $[a, b]$, then we define $\int_a^b f(x)\ dx = \stp{S_a^b(f, \Delta x)}$ for some positive infinitesimal $\Delta x$.

\begin{lemma}
    Let $f: \reals \to \reals$ and $[a, b] \subseteq \reals$. Given any two partitions $P_1$ and $P_2$ of $[a, b]$, we find $L_a^b(f, P_1) \leq U_a^b(f, P_2)$.
\end{lemma}

\begin{proof}[Proof hinted at in \cite{goldblatt1998}]
    
\end{proof}

\begin{corollary}\label{lowerSumsLessThanUpperSumsWithNumbers}
    If $0 < \Delta x_1, \Delta x_2$, then $L_a^b(f, \Delta x_1) \leq U_a^b(f, \Delta x_2)$.
\end{corollary}

\begin{thm}
    If $f$ is integrable on $[a, b]$, then $\int_a^b f(x)dx$ is well-defined.
\end{thm}

\begin{proof}[Proof adapted heavily from \protect{\cite[ch.~9.2]{goldblatt1998}}]
    The issue here is that $\int_a^b f(x)\ dx = \stp{S_a^b(f, \Delta x)}$ might depend on our choice of positive infintiesimal $\Delta x$. Let $\Delta x_1$ and $\Delta x_2$ be two positive infinitesimals. We want to show that $S_a^b(f, \Delta x_1) \simeq S_a^b(f, \Delta x_2)$.

    Note that in the reals, $(\forall \Delta x)(S_a^b(f, \Delta x) \leq U_a^b(f, \Delta x))$. This is because $\sum_{i=0}^n f(a + (i-1)\frac{b-a}{\Delta x})\Delta x \leq \sum_{i=0}^n M_i \Delta x$, as $M_i$ is a maximum of $f$ on $[a + (i-1)\frac{b-a}{\Delta x},  + i\frac{b-a}{\Delta x}]$ and so $f(a + (i-1)\frac{b-a}{\Delta x}) \leq M_i$. We can transfer this statement to the hyperreals to conclude that $S_a^b(f, \Delta x_1) \leq U_a^b(f, \Delta x_1)$, and by a similar line of reasoning we conclude $L_a^b(f, \Delta x_1) \leq S_a^b(f, \Delta x_1)$. Since $L_a^b(f, \Delta x_1) \simeq U_a^b(f, \Delta x_1)$, this implies $S_a^b(f, \Delta x_1) \simeq U_a^b(f, \Delta x_1)$ as well. All of this equally applies to $\Delta x_2$, of course.

    Let $L_1 = L_a^b(f, \Delta x_1)$, $L_2 = L_a^b(f, \Delta x_2)$, and similarly for $U_1$ and $U_2$. We know by \autoref{lowerSumsLessThanUpperSumsWithNumbers} that $L_1, L_2 \leq U_1, U_2$, and so the possible orderings are $L_1 \leq L_2 \leq U_1 \leq U_2$, $L_1 \leq L_2 \leq U_2 \leq U_1$, or either of those with the indices swapped. In any case, the fact that $L_1 \simeq U_1$ and $L_2 \simeq U_2$ implies $L_1 \simeq L_2 \simeq U_1 \simeq U_2$. But we have $S_a^b(f, \Delta x_1) \simeq U_1 \simeq U_2 \simeq S_a^b(f, \Delta x_2)$, so we are done.
\end{proof}

\subsection{Integrals as Hyperfinite Sums}
Now, let $f: [a, b] \to \reals$ be an integrable function. To take $\int_a^b f(x)dx$, we want to divide $[a, b]$ into infinitely many segments of infinitesimal width, and then add up the area of the rectangles above or below those segments. Hyperfinite sums give us a way to do this.

To divide $[a, b]$ into infinitely many segments of infinitesimal width, we will construct a hyperfinite partition where the segments are of infinitesimal width $dx > 0$. Say $dx = [\langle \Delta x_1, \Delta x_2, \ldots \rangle]$. Let $P_n \cup \{b\}$ be partition of $[a, b]$ into segments of width $\Delta x_n$ (plus a final``remainder'' segment of length $\leq \Delta x_n$). So
\[
P_n = \left\{a + k\Delta x_n \sthat 0 \leq k < \frac{b-a}{\Delta x_n},\ k \in \nats\right\}.
\]
Let $N_n$ denote $|P_n| = \frac{b-a}{\Delta x_n}$, and let $r_n$ denote the length of the ``remainder'' segment $r_n = b - (a + \Delta x_n(N_n-1))$. Notice $r_n \leq \Delta x_n$. Then, the left Riemann sum of $f$ on $[a, b]$ with partition $P_n \cup \{b\}$ is 
