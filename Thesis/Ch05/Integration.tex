\section{Integration}
\subsection{Hyperfinite Sets}
\begin{defn}
    An internal set $A = [A_n]$ is \textit{hyperfinite} if every $A_n$ is finite.
\end{defn}

The hyperfinite sets are ``internally finite.'' They share a lot of properties with finite sets.

\begin{thm}\label{HyperfiniteTransfer}
    If $\varphi(A)$ holds for every finite $A \subseteq \reals$, then $\hr{\varphi}(X)$ holds for every hyperfinite $X \subseteq \hreals$.
\end{thm}

\begin{proof}
    Say $X = [A_n]$, with each $A_n$ finite. Then $[[\varphi(A_n)]] = \nats$, and so by transfer $\hr{\varphi}(X)$.
\end{proof}

This lets us easily get a lot of nice properties about hyperfinite sets. For instance, $(\exists x \in A)(\forall y \in A) (x \geq y)$ ensures that every hyperfinite set has a maximum element.

\subsection{Hyperfinite Sums}
For any finite set $A_n$ and function $f_n: \reals \to \reals$, we can easily define the sum $\sum_{x \in A_n} f(x)$. This is, after all, just a sum of a finite collection of numbers. Using this, however, we can easily extend our summation to hyperfinite sets:

\begin{defn}
    If $A = [A_n]$ is a hyperfinite set, and $f = [f_n]$ is an internal function, we define the \textit{hyperfinite sum}
    \[\sum_{x \in [A_n]} f(x) = \left[\ \ \sum_{\mathclap{x \in A_n}} f_n(x) \:\right]\] %Why won't this space properly?
\end{defn}

We'll see this general form used later for integration, but for now we will focus on the type of hyperfinite sums that corresponds to series in standard calculus. Fist, some notation. Let $\{0\ldots n\}$ denote $\{0, 1, 2, \ldots, n\}$ for any $n \in \nats$. Then, let $\mathscr{N} = [\langle 0, 1, 2, \ldots \rangle]$, and let $\{0\ldots\mathscr{N}\}$ denote $[\{0\ldots n\}]$.

For a sequence $a_i: \nats \to \reals$, we have $\sum_{i \in \{0\ldots n\}} a_i = \sum_{i = 0}^n a_i$. So we denote $\sum_{i \in \{0\ldots \mathscr{N}\}} a_i = \sum_{i = 0}^\mathscr{N} a_i = \left[\sum_{i=0}^n a_i\right]$.