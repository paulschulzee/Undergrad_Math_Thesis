\section{Ultrafilters, Ultrapowers, \texorpdfstring{$\hreals$}{*R}}\label{sec:Ultrapowers}
\subsection{The Ultrafilter}
The idea behind the \textit{ultrafilter} $\mathcal{F} \subseteq \mathcal{P}(\nats)$ is that it should include all of the ``very large'' subsets of $\nats$. This is so that we can define an equivalence relation $\equiv$ on sequences of real numbers, so that $\langle r_n \rangle \equiv \langle s_n \rangle$ when they are the same ``almost everywhere,'' i.e. when $\{n \in \nats \sthat r_n = s_n \} \in \mathcal{F}$. We will use the following definition.

\begin{defn}
    $\mathcal{F} \subseteq \nats$ is a (non-principal) ultrafilter on $\nats$:
    \begin{enumerate}
        \item whenever $A, B \in \mathcal{F}$, we have $A \cap B \in \mathcal{F}$, and
        \item whenever $A \in \mathcal{F}$ and $A \subseteq B$, we have $B \in \mathcal{F}$, and
        \item for any $A \subseteq \nats$, \textit{either} $A$ or $A^c = \nats - A$ are in $\mathcal{F}$, and
        \item no finite set is in $\mathcal{F}$.
    \end{enumerate}
\end{defn}

A subset of $\mathcal{P}(\nats)$ that satisfies 1 and 2 is a \textit{filter}. For instance, $\mathcal{P}(\nats)$ is a filter on $\nats$, as is $\emptyset$. A \textit{proper} filter is one that does not contain $\emptyset$ (as if $\emptyset \in \mathcal{F}$, then $\mathcal{P}(\nats) = \mathcal{F}$ by 2). Strictly speaking, any proper filter that satisfies 3 is an ultrafilter, but we will use ``ultrafilter'' to refer only to filters that satisfy 4 as well. 

Call a set \textit{cofinite} if its complement is finite. Every cofinite subset of $\nats$ is in $\mathcal{F}$, by 3 and 4. 

If we think of $\mathcal{F}$ as being the subsets of $\nats$ that include ``almost all'' of the natural numbers, then these make intuitive sense. If $A$ and $B$ both include ``almost all'' of the natural numbers, then surely $A \cap B$ does too---``basically no'' elements are in $A - B$ or $B - A$ since ``basically no'' elements are outside $A$ or $B$. If $A$ includes ``almost all'' of $\nats$, and $A \subseteq B$, then surely $B$ includes ``almost all'' of $\nats$ too. Etc. Note also that by a simple induction on 1, $\mathcal{F}$ is closed under finite intersections.

Of course, there are some unintuitive things about $\mathcal{F}$. By 3, it contains either the set of even numbers $2\nats$ or the set of odd numbers $2\nats + 1$, but not both.

\begin{thm}
    There is at least one ultrafilter $\mathcal{F}$ on $\nats$.
\end{thm}

\begin{proof}[Proof adapted from \protect{\cite[20-21]{goldblatt1998}}]
    Let $\mathcal{F}^{co} \subset \mathcal{P}(\nats)$ denote the collection of cofinite subsets of $\nats$. Let $P$ denote the collection of all proper filters on $\nats$ that include $\mathcal{F}^{co}$. Since $\mathcal{F}^{co}$ is itself a filter, $P \neq \emptyset$. 

    $P$ is partially ordered by $\subseteq$. Our approach is to apply Zorn's Lemma. Let $T \subset P$ be totally ordered by $\subseteq$. Then $\cup T$ is clearly an upper bound of $T$, but we need to show that $\cup T$ is a filter (this is stated but not proven as \cite[Example~2.4(4)]{goldblatt1998}). If $A, B \in \cup T$, then $A \in T_1$ and $B \in T_2$ for some $T_1, T_2 \in T$. Since $T$ is totally ordered by $\subseteq$, either $T_1 \subseteq T_2$ or $T_2 \subseteq T_1$. If $T_1 \subseteq T_2$, then $A \in T_2$, and so since $T_2$ is a filter $A \cap B \in T_2$ and thus $A \cap B \in \cup T$. The proof is similar if $T_2 \subseteq T_1$. Similarly, if $A \in \cup T$ and $A \subseteq B$, then $A \in T_1 \in \cup T$ and so $B \in T_1$ since $T_1$ is  filter. So $B \in \cup T$.

    So any totally ordered subset $T \subset P$ has a maximal element. By Zorn's Lemma, $P$ has a maximal element---call it $\mathcal{F}$. We want to show $\mathcal{F}$ is an ultrafilter (this is \cite[Exercise~2.5(6)]{goldblatt1998}). By the definition of $P$, $\mathcal{F}$ is a proper filter. Since $\mathcal{F}^{co} \subseteq \mathcal{F}$, if we can show that for every $A \subseteq \nats$ either $A$ or $A^c$ is in $\mathcal{F}$ then we will be done.

    Take $A \subseteq \nats$. We cannot have $A \in \mathcal{F}$ and $A^c \in \mathcal{F}$, for then we'd have $A \cap A^c = \emptyset \in \mathcal{F}$, which is impossible since $\mathcal{F}$ is proper. Now, assume for a contradiction that $A \notin \mathcal{F}$ and $A^c \notin \mathcal{F}$. We will show that $\mathcal{F}$ can be extended by adding $A$ or $A^c$, showing that $\mathcal{F}$ is not a maximal element of $P$ and obtaining our contradiction. Then either $A \in \mathcal{F}$ or $A^c \in \mathcal{F}$, and since this will hold for any $A$ we will be done.

    Let $\mathcal{F}'$ be the filter obtained by adding to $\mathcal{F}$ the the intersection $A \cap B$ for any $B \in \mathcal{F}$, and any superset of $A \cap B$ (this technique adapted from \cite[Appendix~A]{henle1979}). Note that $A = A \cap \nats$ and $\nats \in \mathcal{F}$, so $A \in \mathcal{F}'$. We will show that $\mathcal{F}' \in P$. First, for any $B \in \mathcal{F}$, we have $B \nsubseteq A^c$ (as $A^c \notin \mathcal{F}$) and so $A \cap B \neq \emptyset$, showing $\mathcal{F}'$ is proper. Next, if $X, Y \in \mathcal{F}' - \mathcal{F}$, where $A \cap B \subseteq X$ and $A \cap C \subseteq Y$, then $(A \cap B) \cap (A \cap C) = A \cap (B \cap C) \subseteq X \cap Y$, with $B \cap C \in \mathcal{F}$, so $X \cap Y \in \mathcal{F}'$. A similar proof shows that for any $X \in \mathcal{F}$ and $Y \in \mathcal{F}' - \mathcal{F}$, $X \cap Y = Y \cap X \in \mathcal{F}'$. Similarly, if $X \in \mathcal{F}'-\mathcal{F}$ and $X \subseteq Y$, where $A \cap B \subseteq X$, then $A \cap B \subseteq Y$ and so $Y \in \mathcal{F}'$. So $\mathcal{F}'$ is a proper filter, hence $\mathcal{F} \subset \mathcal{F}' \in P$, violating our assumption that $\mathcal{F}$ is maximal in $(P, \subseteq)$. So we have our contradiction and we are done.
\end{proof}

\begin{thm}\label{ultrafilterDisjointUnion}
    Let $\mathcal{F}$ be an ultrafilter and $\{A_1, \ldots, A_n\}$ a finite collection of pairwise disjoint sets such that 
    \[ A_1 \cup \cdots \cup A_n \in \mathcal{F} \]
    Then $A_i \in \mathcal{F}$ for \textit{exactly one} $i$ such that $1 \leq i \leq n$.
\end{thm}

\begin{proof}[Proof \protect{\cite[Exercise~2.5(4)]{goldblatt1998}}]
    At most one of the $A_i$'s can be in $\mathcal{F}$, since if $A_i, A_j \in \mathcal{F}$ when $i \neq j$ then we'd have $A_i \cap A_j = \emptyset \in \mathcal{F}$, a contradiction.

    Assume for a contradiction that $A_i \notin \mathcal{F}$ for each $i$. Then $A_i^c \in \mathcal{F}$ for each $i$, and so since $\mathcal{F}$ is closed under finite intersections we find
    \[ \bigcap_{i=1}^n A_i^c \in \mathcal{F}. \]
    But then we find
    \[ \bigcup_{i=1}^n A_i  = \left(\bigcap_{i=1}^n A_i^c\right)^c \notin \mathcal{F}, \]
    a contradiction.
\end{proof}

\subsection{The Ultrapower}
Let $\reals^\nats$ denote the set of sequences in $\reals$. We will denote a member $r = \langle r_1, r_2, r_3, \ldots \rangle$ of $\reals^\nats$ by $\langle r_n \rangle$. We can define operations termwise addition $\oplus$ and termwise multiplication $\odot$ on $\reals^\nats$ by $\langle r_n \rangle \oplus \langle s_n \rangle = \langle r_n + s_n \rangle$ and $\langle r_n \rangle \odot \langle s_n \rangle = \langle r_n \cdot s_n \rangle$, giving us a commutative ring $\left(\reals^\nats, \oplus, \odot\right)$.

Now, let $\equiv$ denote the relation such that $\langle r_n \rangle \equiv \langle s_n \rangle$ iff $\{n \in \nats \sthat r_n = s_n\} \in \mathcal{F}$. We will write $[[r_n = s_n]]$ to denote $\{n \in \nats \sthat r_n = s_n\}$. When $\langle r_n \rangle = \langle s_n \rangle$, we will write that $r_n = s_n$ $\mathcal{F}$-almost everywhere.

\begin{thm}
    $\equiv$ is an equiavlence relation on $\reals^\nats$
\end{thm}

\begin{proof}[Proof \protect{\cite[Exercise~3.3(1)]{goldblatt1998}}]
    Clearly $\equiv$ is reflexive, since $[[r_n = r_n]] = \nats \in \mathcal{F}$ so $\langle r_n \rangle \equiv \langle r_n \rangle$.
    
    Similarly, $\equiv$ is symmetric because $[[r_n = s_n]] = [[s_n = r_n]]$, and so $[[r_n = s_n]] \in \mathcal{F}$ iff $[[s_n = r_n]] \in \mathcal{F}$, and so $\langle r_n \rangle \equiv \langle s_n \rangle$ iff $\langle s_n \rangle \equiv \langle r_n \rangle$.

    Finally, $\equiv$ is transitive. Say $\langle r_n \rangle \equiv \langle s_n \rangle \equiv \langle t_n \rangle$. Then $[[r_n = s_n]] \in \mathcal{F}$ and $[[s_n = t_n]] \in \mathcal{F}$. Whenever $r_n = s_n$ and $s_n = t_n$, we have $r_n = t_n$, and so $[[r_n = s_n]] \cap [[s_n = t_n]] \subseteq [[r_n = t_n]]$. Since $\mathcal{F}$ is a filter, this implies $[[r_n = t_n]] \in \mathcal{F}$, and so $\langle r_n \rangle \equiv \langle t_n \rangle$.
\end{proof}

Now, we will form equivalence classes in $\reals^\nats$ based on this equivalence relation. Note that $[[r_n = s_n]] \in \mathcal{F}$ iff $[[r_n - s_n = 0]] \in \mathcal{F}$, and so if $I = \{\langle r_n \rangle \in \reals^\nats \sthat [[r_n = 0]] \in \mathcal{F}\}$ then $\langle r_n \rangle \equiv \langle s_n \rangle$ iff $\langle r_n \rangle \ominus \langle s_n \rangle \in I$. $I$ is closed under subtraction: if $\langle r_n \rangle, \langle s_n \rangle \in I$, then $[[r_n = 0]] \cap [[s_n = 0]] \subseteq [[r_n - s_n = 0]] \in \mathcal{F}$, so $\langle r_n \rangle - \langle s_n \rangle \in I$. Furthermore, for any $\langle r_n \rangle \in \reals^\nats$ and $\langle s_n \rangle \in I$, we have $[[r_n \cdot s_n = 0]] \supseteq [[s_n = 0]] \in \mathcal{F}$ and so $[[r_n \cdot s_n = 0]] \in \mathcal{F}$ and $\langle r_n \rangle \cdot \langle s_n \rangle \in I$. We conclude that $I$ is an ideal. Then, we define the \textit{hyperreals}
\[ \hreals = \reals^\nats / I. \]
We denote the equivalence class of an element $r = \langle r_n \rangle$ by $[r]$, $[\langle r_n \rangle]$, or just $[r_n]$ (ommitting the angled brackets).

One might reasonably worry that we chose no particular ultrafilter $\mathcal{F}$ on $\nats$, and so $\hreals$ might not be unique. We can prove that $\hreals$ is unique if we assume the continuum hypothesis---otherwise, the uniqueness of $\hreals$ is indeterminate in ZFC \cite[33]{goldblatt1998}.

\subsection{Extensions}\label{sec:extensions}

For any function $f: \reals \to \reals$, we define $\hr{f}: \hreals \to \hreals$ by $\hr{f}([r_n]) = [f(r_n)]$. For example, if $f(x) = x^2$, then $f([\langle 1, 2, 3, \ldots \rangle]) = [\langle 1, 4, 9, 16, \ldots \rangle]$. $\hr{f}$ is well defined, as if $[r_n] = [s_n]$ then $[[r_n = s_n]] \in \mathcal{F}$, and so we have $[[r_n = s_n]] \subseteq [[f(r_n) = f(s_n)]] \in \mathcal{F}$ and so $\langle f(r_n) \rangle \equiv \langle f(s_n) \rangle$, i.e. $\hr{f}([r_n]) = [f(r_n)] = [f(s_n)] = \hr{f}([s_n])$.

Similarly, for any $k$-ary relation on the reals $R_k \subseteq \reals^k$, we define $\hr{R}_k \subseteq \hreals^k$ by setting $\hr{R}_k([r_n^1], [r_n^2], \ldots, [r_n^k])$ if $[[R_k(r_n^1, r_n^2, \ldots, r_n^k)]] \in \mathcal{F}$. This is well defined because if $[s_n^i] = [r_n^i]$ for every $1 \leq i \leq k$, then $[[s_n^i = r_n^i]] \in \mathcal{F}$ for every $i$, and so we find
\[
\left([[R_k(r_n^1, \ldots, r_n^k)]] \cap \bigcap_{i = 1}^k [[r_n^i = s_n^i]]\right) \subseteq [[R_k(s_n^1, \ldots, s_n^k)]] \in \mathcal{F}
\]
since the right side is a finite intersection.

$1$-ary relations are subsets, and for them we will use $x \in \hr{R}$ in place of $R(x)$. For instance, $[r_n] \in \hnats$ iff $[[r_n \in \nats]] \in \mathcal{F}$. The set $\hnats$ is the set of \textit{hypernaturals}.

We will use symbols such as $<$, $\leq$, $+$, $\cdot$, etc. to refer to their own extensions in the hyperreals. For instance, $[r_n] \leq [s_n]$ iff $[[r_n \leq s_n]] \in \mathcal{F}$, and $[r_n] + [s_n] = [r_n + s_n]$.

A technique we will use frequently is choosing a representative of an equivalence class to meet certain conditions. Say, for instance, we know $[r_n] \in \hnats$. Then $[[r_n \in \nats]] \in \mathcal{F}$. Now, let $s_n = r_n$ for all $n \in [[r_n \in \nats]]$, and $s_n = 0$ elsewhere. Then $[[r_n \in \nats]] \subseteq [[r_n = s_n]] \in \mathcal{F}$, so $[r_n] = [s_n]$, and we have $s_n \in \nats$ for \textit{all} $n$. In other words, whenver a condition is true $\mathcal{F}$-almost everywhere for every member of an equivalence class, we can (usually) pick a member where it is true \textit{actually} everywhere. In proofs, we will say something like ``$[r_n] \in \hnats$, and so we can assume $r_n \in \nats$ for all $n$.''

\subsection{\texorpdfstring{$\reals$}{R} in \texorpdfstring{$\hreals$}{*R}}
For any real number $b \in \reals$, we have $[b] = [\langle b, b, b, \ldots \rangle] \in \hreals$. If $b, c \in \reals$, then $[b] + [c] = [b + c]$, $[b] \leq [c]$ iff $b \leq c$, etc. In these cases we will usually drop the parenthesis and just write $b \in \hreals$. For instance, we might write $[r_n] < 5$ to indicate $[r_n] < [5]$.

\subsection{Internal Sets \& Functions}
Not every subset of or function on $\hreals$ is represented by the extension of a subset of or function on $\reals$.

\begin{thm}
    Let $S \subseteq \reals$. Then $\hr{S} - \reals \neq \emptyset$.
\end{thm}

\begin{proof}[Proof from \protect{\cite[Theorem~3.9.1]{goldblatt1998}}]
    The theorem here is relying a bit on an abuse of notation, using $\reals$ in the first case to mean the real numbers and in the second case to mean the ``copy'' of the real numbers in $\hreals$.

    Let $s_1, s_2, \ldots \in S$ be a pairwise distinct sequence of elements of $S$. Then clearly $[s_n] \in \hr{S}$ as $[[s_n \in S]] = \nats \in \mathcal{F}$. But for any $x \in \reals$, we find $[[s_n = x]]$ is either $\emptyset$ or a singleton, since the $s_n$ are pairwise distinct. So $[[s_n = x]] \notin \mathcal{F}$, i.e. $[s_n] \neq x$.
\end{proof}

\begin{thm}
    If $A$ is finite, then $\hr{A} = A$.
\end{thm}

\begin{proof}[Proof \protect{\cite[Exercise~3.10(1)]{goldblatt1998}}]
    Let $[r_n] \in \hr{A}$. Then $[[r_n \in A]] \in \mathcal{F}$, and we have 
    \[ \bigcup_{b \in A} [[r_n = b]] = [[r_n \in A]] \in \mathcal{F}. \]
    Note that the $[[r_n = b]]$'s are pairwise disjoint. Then, by \autoref{ultrafilterDisjointUnion}, we know that $[[r_n = b]] \in \mathcal{F}$ for \textit{exactly one} $b$, and so we conclude $[r_n] = b$. So every element of $\hr{A}$ is an element of $A$. Furthermore, for any $b \in A$, we have $[[b \in A]] = \nats \in \mathcal{F}$, and so $b \in \hr{A}$. So by double inclusion, $\hr{A} = A$. 
\end{proof}

So the extension of any finite subset of $\reals$ is finite, and the extension of any infinite subset contains ``nonstandard'' elements not in $\reals$. So if we consider $\nats$ as a subset of $\hreals$, we find that it isn't the extension of anything: it's infinite, but it doesn't contain any non-real elements.

So we can't use extensions to work with every possible subset of (or function on) $\hreals$. What we can do, though, is find ``well-behaved'' subsets or functions that we can extend our methods to. These well-behaved subsets and functions are called \textit{internal}.

\begin{defn}
    If $A_1, A_2, A_3, \ldots$ is a sequence of subsets of $\reals$, we define the \textit{internal subset} $A \subseteq \hreals$, denoted $A = [A_n]$, by
    \[ [r_n] \in [A_n] \text{ iff } [[r_n \in A_n]] \in \mathcal{F}. \]
\end{defn}

Internal set membership is well-defined. If $[r_n] \in [A_n]$, then $[[r_n \in A_n]] \in \mathcal{F}$, so if $[r_n] = [s_n]$ then $[[r_n \in A_n]] \cap [[r_n = s_n]] \subseteq [[s_n \in A_n]] \in \mathcal{F}$ and so $[s_n] \in [A_n]$. Any finite set of hyperreals is internal, as if $X = \{[r_n^1], \ldots, [r_n^k]\}$ then $X = [\{r_n^1, \ldots, r_n^k\}]$ \cite[126]{goldblatt1998}. A finite set of hyperreals is only the extension of a set of reals if it is itself also a set of reals, so any finite set of hyperreals that contains non-real elements is internal but not the extension of a set of reals. Internal functions are defined similarly to internal sets:

\begin{defn}
    If $f_1, f_2, f_3, \ldots$ is a sequence of functions $f_n: D_n \to \reals$, we define the \textit{internal function} $f: [D_n] \to \hreals$, denoted $f = [f_n]$, by
    \[ [f_n]([r_n]) = [f_n(r_n)]. \]
\end{defn}



