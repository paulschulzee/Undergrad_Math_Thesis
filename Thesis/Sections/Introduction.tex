\section*{Introduction}
The idea of the \textit{infinitesimal}, an infinitely small number that is still distinct from $0$, has been around since the very beginning of calculus. Not only did Leibniz formulate his calculus in terms of infinitesimals, but they were used calculus education long after his death as well. Of course at the time infinitesimal theories were ill-formed, which made it difficult for mathematicians or students to go beyond simple derivatives and integrals \cite{henle1979}. It wasn't until the early 1960's when Abraham Robinson managed to formalize infinitesimal calculus, creating the field of non-standard analysis. Today non-standard analysis goes beyond simple calculus on the reals, but this paper will focus on the rigorous development of infinitesimal calculus, assuming no prior knowledge. Abraham Robinson's formulation of the infinitesimal calculus relied on the \textit{compactness theorem} in mathematical logic---it is taught in MATH 499. We will instead use the more popular \textit{ultrapower} formulation, which has the benefit of requiring less mathematical background. So much so that Henle \& Kleinberg, in \cite{henle1979}, attempted to write a book aimed at those learning calculus for the first time.

In \autoref{sec:Ultrapowers}, we will go through the development of the \textit{hyperreals} $\hreals$ as an \textit{ultrapower} of $\reals$. The basic idea is to take the ring of countable sequences of real numbers, $\reals^\nats$, and form equivalence classes. Two sequences will be in the same equivalence class if they are the same ``almost everywhere''---a concept we will define rigorously using an \textit{ultrafilter} on $\nats$. $\hreals$ will end up being a set that contains all the real numbers within it, as well as a plethora of infinitesimal elements.

In \autoref{sec:FOLTransfer}, we will develop a formal mathematical language. This language will be able to say much less than English, but we will prove the \textit{transfer} principle, that any sentence of our formal language is true in $\reals$ if and only if it is true in $\hreals$. This will give us a powerful tool to learn about $\hreals$.

In \autoref{sec:StructureOf*R}, we will explore the structure of $\hreals$---what hyperreals are there, how do they relate to the real numbers, what can we say about subsets and functions on $\hreals$, etc. The concepts we cover in this section will be largely utilitarian, enabling our discussion in the final two sections.

In \autoref{sec:Differentiation} and \autoref{sec:Integration}, we will develop differential and integral calculus. Our basic approach will be to take some useful fact about $\reals$, transfer it to $\hreals$, use that fact to prove some result we want in $\hreals$, and then transfer that result back to $\reals$. In this way, we will prove things about $\reals$ by going ``through'' $\hreals$. Additionally, we will develop definitions of the derivative and integral that hew more closely to our intuition about what they are, as opposed to unwieldy $\epsilon$-$\delta$ definitions of standard calculus. \autoref{sec:expSec}, about the exponential function $\exp$, was developed independently without the aid of any texts (although it has been done before).

Theorems, lemmas, and corollaries have all been credited to Goldblatt \cite{goldblatt1998} or Henle \& Kleinberg \cite{henle1979}, the two texts referenced for this paper, where appropriate. If a theorem, lemma, or corollary is not cited, it to my knowledge did not appear in either text. (There is at least one example in the paper of a theorem that is not stated in the most relevant section of \cite{goldblatt1998}, but is a consequence of a later theorem, so there may be more instances of that that I am unaware of.) Definitions are from \cite{goldblatt1998}, with two exceptions. Firstly, the language used in \autoref{sec:FOLTransfer} has been modified for simplicity (as we will not be doing non-standard analysis on anything but $\reals$), and the treatment of set symbols has been made rigorous. Secondly, the definition of $\exp = \sum_{i=0}^\infty \frac{x^i}{i!}$ is not listed in either text, although it is hardly original. Proofs are frequently listed as ``adapted from'' one of the texts, meaning that they were modified in some way to be done with the tools already defined in the paper or otherwise streamlined. The proof of a restricted \L o\'s's Theorem in \autoref{sec:FOLTransfer} got the core approach of inducting on wffs from \cite{henle1979}, but has been heavily modified to work with the language presented and give a formal treatment of set symbols, which in turn enables easy proofs of several theorems that appear in \cite{goldblatt1998}.
